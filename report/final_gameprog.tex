\documentclass[11pt,a4paper]{article}
\usepackage[utf8]{inputenc}
\usepackage{amsmath}
\usepackage{amsfonts}
\usepackage{amssymb}
\usepackage{graphicx}
\author{Frank Smit \& Sander Latour}
\title{A game of politics}
\begin{document}
\maketitle

\begin{abstract}

\end{abstract}

\section{Introduction}
Ever wanted to be the president of a country? The game; A game of politics, allows the gamer to crawl into the skin of a presidential character during the prehistoric era...

\section{The game}
The idea of the game is that during the prehistoric era you, as the gamer, are put forward as the leader of your country. To help you with making decisions which should improve the living conditions of your people, five members, each of one tribe, will motion to subjects they think are important. The way the gamer rules the country is up to the gamer. He, for example, could choose to rule in a democratic manner or choose to rule the country as a dictator. However, the goal of the game is to be a president until you retire, so as a gamer you would like the people of you country to support you.

So how does this work? First of all when the game starts the gamer posses 1\% of the power. The remaining power is divided over the different tribes, the bigger the tribe the more power it posses. In every round, the members propose to the gamer their wishes. If the gamer fulfills a wish of a tribe, the relationship between the gamer and that tribe will improve. This relationship could such enhance that the tribe decides to support you unconditionally. This means that they will not propose anymore wishes. This means that the gamer's power will increase with the amount of power the tribe had. At the end of every turn, the player should have 50 \% of the people (e.g. power) supporting him. If this is not the case, the game is finished. 

However there is a small catch, the relationship between the gamer and the tribe could also deteriorate by not considering the ideals of the tribe. For example, one tribe asks you to build more farms so that there is more food, then another tribe, who for example wants more safety will be less happy when you grant the wish of the first tribe. This because, when building farms there are less people available to be standing on watch. All these positive and negative consequences affect the emotional state of the tribe. So when a tribe supports the gamer unconditionally, he should be aware of the ideals of the tribe because when the gamer grants a wish which is bad for the ideals of the supporting tribe, the emotional state of that tribe will get worse. This could continue until the tribe decides to stop supporting you unconditionally and your power is weakened. At that point the tribe will ask you again to grant his wishes.

So back to the point made about the different forms of government, it is possible to be a dictator by getting more than 50 \% of the people to unconditionally support you. And you could have a democratic form by keeping the power divided between the different tribes.

\section{Tribes}
The tribes in the game are modeled ask AI agents, where they differ are only in their ideals and their level of patience (the more patient an agent is the more he will allow you to disobey his wishes). The agent have five levels of emotional states, which are connected in a linear fashion. These states are, ...... The actions that an agents could take in a certain stage of the game is determined by means of a behavior tree (reference...). The way these agents are modeled and how actions are chosen will be explained in this section. 

\subsection{Modeling Agents}

\end{document}